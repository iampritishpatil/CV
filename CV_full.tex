%!TEX TS-program = xelatex

\documentclass[11pt]{friggeri-cv}%,print
%\addbibresource{bibliography.bib}


\usepackage{enumitem}

\usepackage{multicol}

\setlist[itemize]{leftmargin=*}

\setlength{\columnsep}{2cm}
\begin{document}
\header{pritish } {patil} 
        {\href{mailto:iampritishpatil@gmail.com}{iampritishpatil@gmail.com}\quad  
    \href{mailto:ugpatil@ug.iisc.in}{ugpatil@ug.iisc.in}\quad{ +91-8861-557-553}
}



\section{interests}

Theoretical Neuroscience, Computational Neuroscience, Stochastic Modeling, Numerical Methods, Systems Biology, Stochastic Differential Equations, Spatial Dynamics, Applied Mathematics in Biology.


\section{education}

\begin{entrylist}
  \entryy
    {2012--2016}
    {Bachelor of Science {\normalfont Biology Major with Mathematics Minor }}
    {Indian Institute of Science, Bangalore}
    { CGPA 6.6/8.0  (After 6 semesters)}
  \entryy
    {2012}
    {12th Grade {\normalfont Science Stream}}
  {KVN Naik College, Nashik}
  { 79.83\%}
  \entryy
    {2010}
    {10th Grade     {\normalfont Matriculation}}
    {JDC Bytco High School, Nashik}
    {86.16\%}

\end{entrylist}


\section{major achievements}
\begin{entrylist}
  \entryyy
    {2012}
    { {\normalfont Silver Medal@}International Biology Olympiad     }
    {Singapore, Singapore}

  \entryyy
    {2011}
    {  {\normalfont Silver Medal@}International Biology Olympiad}
    {Taipei, Taiwan}
    
  \entryyy
    {2010}
    { {\normalfont Silver Medal@}International Astronomy Olympiad }
    {Crimea, Ukraine}



\end{entrylist}

\section{research experience}

\begin{entrylist}
  \entry
    {2015}
    {Making a realistic model CA1 Pyramidal Neuron in MOOSE}
    {NCBS, Bangalore}
    {Guide : Dr. Upi Bhalla, NCBS, Bangalore}
    {Coming up with a distrubution ion channels for the CA1 Pyramidal neurons which has realistic behavior for different morphologies. All coding in MOOSE}
\end{entrylist}
\begin{entrylist}
  \entry
    {2014}
    {Finding network topologies which show adaptation response}
    {NCBS, Bangalore}
    {Guide : Dr. Sandeep Krishna, NCBS, Bangalore}
    {Modelled a general three node gene/protein network using a system of differential equations and simulated it. The aim was to find the topologies which show the adaptation response. Programming was done in C. Used variable step-size 4th order Runge-Kutta routine to solve the system of differential equations.}
\end{entrylist}
\begin{entrylist}

  \entry
    {2013}
    {Modelling of High Energy Cosmic Ray Spectrum}
    {HBCSE, Mumbai}
    {Guide : Prof. Mayank Vahia, TIFR, Mumbai}
    {Explored the effect of magnetic field on cosmic rays produced inside galaxies and proposed an explanation for the features seen in the cosmic ray spectrum. Tried to explain galactic X-Ray halos using these cosmic rays. Matlab and C were used.
    }
\end{entrylist}
\begin{entrylist}
    
  \entry
    {2013}
    {Lab techniques for isolation and purification of proteins}
    {IISc Bangalore}
    {
    Guide : Prof. V. Nagaraja, IISc, Bangalore}
    {Learned various lab techniques like Polyacrylamide Gel Electrophoresis, Ion-exchange Chromatography, Affinity and Immunoaffinity Chromatography, Metal Chelate Affinity Chromatography,  Size-exclusion Chromatography. General techniques in microbiology were also learned.
    }    
\end{entrylist}
\begin{entrylist}

  \entry
    {2012}
    {Constraining Dark Energy Parameters using Supernova-1a data}
    {IISER, Mohali}
    {Guide : Prof H.K. Jassal, IISER Mohali}
    {Understood standard cosmology,  obtained constraints on dark energy parameters of the standard model and evaluated different cosmological models by comparing with SN1A data(Union Supernova Project). Programming and analysis were done in C and MATLAB.
    }
    
\end{entrylist}
\begin{entrylist}


  \entry
    {2012}
    {Karyotyping for screening of chromosomal abnormalities}
    {Genetic Health \& Research Centre, Nasik}
    {
    Guide : Dr. Dnyandeo Chopade, Genetic Health \& Research Centre, Nasik}
    {
    Mastered the basics of Karyotyping. Learned to make karyotypes from blood and from chorionic villi.  Apprenticed for  detection of defects in chromosomes in the karyotypes.
    }
    \end{entrylist}
    \begin{entrylist}


  \entry
    {2011}
    {A stacking analysis of radio properties of photometrically selected quasars}
    {NCRA, Pune}
    {
    Guide : Dr. Yogesh Wadadekar, NCRA, Pune}
    {Analysed the radio properties of 1 million quasars (all the known quasars at that time) found by SDSS photometrically. Correlated the optical data to radio data by doing statistics on radio image stacks of quasars. Programmed in Python using SciPy, NumPy, PyFITS as well as some other astronomy specific Python modules. 
    }
    \end{entrylist}
    \begin{entrylist}
  \entry
    {2010}
    {Effect of metallicity on the evolution of stellar populations}
    {NCRA, Pune}
    {
    Guide : Dr. Yogesh Wadadekar, NCRA, Pune}
    {Studied the effects of changes in metallicity of a nebula upon the evolution of clusters of stars within it. Programming and analysis were done using C and shell script.
    }

    \end{entrylist}
\begin{entrylist}
  \entry
    {2009}
    {Study of Irregularities in the Spiral Structure of M101}
    {HBCSE, Mumbai}
    {
    Guide : Prof. Mayank Vahia, TIFR, Mumbai}
    {Analysed the spiral structure of M101 Pinwheel galaxy, examined the irregularities  and proposed explanations for them. Analysis was done in MATLAB.
    }    
\end{entrylist}

\section{course projects}

\begin{entrylist}
\entry
    {2014}
    {Spatial Dynamics of Sympatric Speciation (Ongoing)}
    {Theoretical and Mathematical Ecology}
    {
    Prof. Vishwesha Guttal, IISc Bangalore}
    {Studied spatial dynamics of sympatric speciation due to disruptive selection. 
    }    
\end{entrylist}
\begin{entrylist}
\entry
    {2014}
    {Leeches:
Animal movements and random walks}
    {Experiment in Ecology}
    {
    Dr. Farah Ishtiaq, IISc Bangalore}
    {Explored how the leeches could be locating their prey in absence of stimulus. Found that the leeches perform a correlated random walk, which emulate a Levy random walk.
    }    
\end{entrylist}
\begin{entrylist}
  \entry
    {2014}
    {Comparing Weiner chaos decomposition and Monte Carlo methods for solving stochastic differential equations.}
    {Introduction to Scientific Computing}
    {
    Prof. S. Raha, IISc Bangalore}
    {Used Weiner Chaos Decomposition and Monte Carlo method to find the solutions of a system of stochastic differential equations numerically. Compared the accuracy  of and the time taken by these methods. Programming and analysis were done in MATLAB.
    }    
\end{entrylist}
\begin{entrylist}
\entry
    {2014}
    {Sexual Selection with a Two Locus Model}
    {Theoretical and Mathematical Ecology}
    {
    Prof. Vishwesha Guttal, IISc Bangalore}
    {Modelled the effects of sexual selection on two loci in haploid and diploid systems analytically, and in more complex cases numerically. Studied various equilibria of the system and determined their stability. Analysed the dynamics of invasion of one genotype by another. Programming and analysis were done in MATLAB.
    }    
\end{entrylist}

\section{programming and computers}
\subsection{Common programming}
C,
 R, 
 Python,
    MATLAB,
{    \lotsoftextfont \LaTeX},
    shell/bash,
    linux.
\subsection{Neuroscience related}
MOOSE, NEURON, BRIAN 

\pagebreak

\begin{multicols}{2}[]

\begin{minipage}{1.1\columnwidth}
\section{relevant courses [grad level]}


\subsection{biology}

\begin{itemize}
\item Topics in Systems Neuroscience
\item Theoretical and Computational Neuroscience
\item Theoretical and Mathematical Ecology
\item Spatial Dynamics in Biology
\item Cellular Neurophysiology
\item Fundamentals of Systems and Cognitive Neuroscience
\item Fundamentals of Molecular and Cellular Neuroscience
\end{itemize}


\subsection{mathematics}
\begin{itemize}

\item Stochastic Processes {\small[martingales and brownian motion]}
\item Probability Theory {\small[measure theoretic]}
\item Measure theory
\item Algebra
\item Topology
\item Linear Algebra
\item Real Analysis
\end{itemize}

\subsection{engineering}
\begin{itemize}
\item Information Theory
\end{itemize}
\vspace{1em}
\section{ relevant introductory courses }
\begin{itemize}
\item Physics (3 courses)
\item Chemistry (3 courses)
\item Mathematics (3 courses)
\item Biology (3 courses)
\end{itemize}



\end{minipage}

\hspace{-1em}
\begin{minipage}{1.05\columnwidth}



\section{relevant courses [undergrad level]}

\subsection{biology}
\begin{itemize}

\item Introductory Structural Biology
\item General Biochemistry
\item Introductory Physiology
\item Developmental Biology
\end{itemize}


\subsection{mathematics}
\begin{itemize}
\item Multivariable Calculus and Complex Variables
\item Elementary Algebra and Number Theory
\item Probability and Statistics
\end{itemize}



\subsection{engineering}
\begin{itemize}
\item Introduction to Scientific Computing
\item Algorithms and Programming
\item Introduction to Electrical and Electronics Engineering
\item Introduction to Material Sciences
\item Introduction to Environmental Sciences
\end{itemize}

\subsection{laboratory courses}
\begin{itemize}
\item Experiments in Biochemistry and Physiology
\item Experiments in Microbiology and Ecology
\item Experiments in Molecular Biophysics
\item Experiments in Neurobiology
\end{itemize}



\end{minipage}

\end{multicols}



\section{other notable achievements}

\begin{entrylist}
  \entryy
    {2011}
    {Selected as a member of Indian team for International Earth Science Olympiad}
    {Modena, Italy}
    {One of top 4 from India to get selected.}
\end{entrylist}


\begin{entrylist}
  \entryy
    {2011 -- 2014}
    {Recipient of KVPY (Kishore Vaigyanik Protsahan Yojana) Scholarship}{}
    {Awarded to the top 200 science students from India each year.}

\end{entrylist}


\begin{entrylist}
  \entryy
    {2009 -- 2011}
    {Recipient of NTSE (National Talent Search Exam) Scholarship}{}
    {Awarded to the top 1000 students from India each year.}
\end{entrylist}


\begin{entrylist}
  \entryy
    {2013}
    {Regionals of ACM International Collegiate Programming Contest}
    {Amrita University Coimbatore}
    {One of the top 389 teams selected from across the country}
\end{entrylist}


\begin{entrylist}
  \entryy
    {2013}
    {Won MIMAMSA, a national inter-college science quiz}
    {IISER, Pune}
    {Qualified for the final quiz from amongst more than 100 teams and WON the 14 hours long quiz.}
\end{entrylist}


\begin{entrylist}
  \entryy
    {2010}
    {The C.L. Bhat Memorial Award for the Best Student}
    {Indian Astronomy Olympiad Camp}
    {Awarded to the best overall performer in the Indian Astronomy Olympiad Camp}

\end{entrylist}

\begin{entrylist}
  \entryy
    {2010-2012}
    {INFOSYS Award for Olympiad Medalists}
    {}
    {Awarded to all the Olympiad medallists every year.}
\end{entrylist}

\begin{entrylist}
  \entryy
    {2010}
    {Rural Electrification Corp. Award for Olympiad Medalists}
    {}
    {Awarded to all the International Olympiad from India.}
\end{entrylist}


\begin{entrylist}
  \entryy
    {2011--2012}
    {Orientation Cum Selection Camp of Indian National Biology Olympiad}
    {HBCSE, Mumbai}
    {In top 35 selected from \approx 6000 from across the country}

\end{entrylist}


\begin{entrylist}
  \entryy
    {2010}
    {Orientation Cum Selection Camp of Indian Junior Science Olympiad}
    {HBCSE, Mumbai}
    {In top 35  selected from \approx 10000 from across the country}

\end{entrylist}

\begin{entrylist}
  \entryy
    {2009--2011}
    {Orientation Cum Selection Camp of Indian Astronomy Olympiad}
    {HBCSE, Mumbai}
    {In top 35 selected from \approx 6000 from across the country}

\end{entrylist}

\begin{entrylist}
  \entryy
    {2011}
    {Selected for Indian National Mathematics Olympiad}{}
    {In top 250 selected from more than 10000 from across the country}

\end{entrylist}

\begin{entrylist}
  \entryy
    {2009--2012}
    {Selected for Indian National Olympiad in Informatics}
    {}
    {In top 250  selected from across the country}
\end{entrylist}


\begin{entrylist}
  \entryy
    {2007--2010}
    {Australian National Chemistry Quiz}
    {}
    {Certificate Of Excellence with Plaque / High Distinction (One from top 7-8 from India every year)}
\end{entrylist}




\section{camps attended}

\begin{entrylist}
  \entryy
    {2014}
    {Physics of Life, NCBS-Simons Annual Monsoon School}
    {NCBS, Bangalore}
    {Topics included: biophysics and soft-matter physics, ranging from aspects molecules to those of cells and tissues; information processing and decision making, at the level of cells or of the brain; stochastic processes in molecules or populations; dynamical systems models of genetic networks or biomechanical systems.}
    
\end{entrylist}


\begin{entrylist}
  \entryy
    {2012 2013}
    {NIUS Astronomy Nurture Camp}
    {HBCSE Mumbai}
    {Worked on various astronomical projects listed above.}
\end{entrylist}


\begin{entrylist}
  \entryy
    {2011, 2012}
    {Vijyoshi Camp}
    {IISc, Bangalore}
    {Similar to Lindau Meet with Noble Laureates for students. For top \approx 600 science students across India }
    
\end{entrylist}

\begin{entrylist}
  \entryy
    {2011, 2012}
    {Biology Olympiad Pre-Departure Training Camp}
    {HBCSE Mumbai}
    {Had training session in theoretical and practical aspects of biology as preparation for the IBO.}
\end{entrylist}

\begin{entrylist}
  \entryy
    {2011, 2012}
    {Biology Olympiad Orientation Cum Selection Camp}
    {HBCSE Mumbai}
    {Lectures and practical training by various faculty. Selection exam in both theory and practicals.}
\end{entrylist}


\begin{entrylist}
  \entryy
    {2011}
    {Earth Science Olympiad Orientation Cum Selection Camp}
    {University of Hyderabad}
    {Lectures and practical training by various faculty. Selection exam in both theory and practicals.}

\end{entrylist}

\begin{entrylist}
  \entryy
    {2010}
    {Astronomy Olympiad Pre-Departure Training Camp}
    {HBCSE Mumbai}
    {Had training session in theoretical and practical aspects of astronomy as preparation for the IAO.}
\end{entrylist}

\begin{entrylist}
  \entryy
    {2009, 2010}
    {NIUS Astronomy Nurture Camp}
    {HBCSE Mumbai}
    {Worked on various astronomical projects listed above.}
\end{entrylist}



\begin{entrylist}
  \entryy
    {2009, 2010}
    {Astronomy Olympiad Orientation Cum Selection Camp}
    {HBCSE Mumbai}
    {Lectures and practical training by various faculty. Selection exam in both theory and practicals.}
\end{entrylist}
\pagebreak
\section{extracurricular activities}

\begin{entrylist}
  \entryy
    {2014}
    {Programming Events Manager for Pravega, the annual college festival}
    {IISc, Bangalore}
    {Reverse coding: given the executable, write the source code \\ Online programming contest: a standard programming contest\\ Connect the dots: a programming treasure hunt, which requires you to solve programming questions to get to the next question}
\end{entrylist}

\begin{entrylist}
  \entryy
    {2013--2014}
    {Convener and founder of Scipher}
    {Bangalore}
    {Scipher is a mock test for the KVPY scholarship exam. We set a model question paper and conducted the model exam across various states in India. 3,500 students took the mock exam Coordinated with ~30 people for setting and designing question paper. Co-ordinated with ~70 people for conduction and supervision of the examination. }
\end{entrylist}


\begin{entrylist}
  \entryy
    {2013}
    {Acted in the play "Photograph 51"}
    {IISc, Bangalore}
    {Character played : Francis Crick}
\end{entrylist}

\begin{entrylist}
  \entryy
    {2012}
    {Acted in and gave sound effects to the play "Safar"}
    {Alliance Française, Bangalore}
    {Character played : Software engineer}
\end{entrylist}




\begin{entrylist}
  \entryyy
    {2013--2014}
    {Active member of Samasya, IISc Math club}
    {IISc, Bangalore}

\end{entrylist}




\begin{entrylist}
  \entryy
    {2012--2013}
    {On the committee of Marathi Mandal}
    {IISc, Bangalore}
    {Group of people in IISc following Maharashtrian traditions and ethnicity}
\end{entrylist}

%\section{experience}
%
%\begin{entrylist}
%  \entry
%    {02–07 2009}
%    {LIP6/CNRS, Paris}
%    {Research Internship.}
%    {\emph{Visualization of complex networks.}}
%  \entry
%    {06–08 2008}
%    {ISCPIF/CNRS, Paris}
%    {Research Internship.}
%    {\emph{Diffusion in the Blogosphere. Happy Flu.}}
%  \entry
%    {06–08 2007}
%    {LIP6/CNRS, Paris}
%    {Research Internship.}
%    {\emph{Kernels in real world networks.}}
%  \entry
%    {07–08 2005}
%    {\href{http://www.kelkoo.com}{Kelkoo.com}}
%    {Summer job.}
%    {\emph{Creation of a keyword generator for Google Adwords.}}
%  \entry
%    {07–08 2004}
%    {\href{http://www.monsieurprix.com}{MonsieurPrix.com}}
%    {Summer job.}
%    {\emph{Development of an e-commerce product indexation spider.}}
%\end{entrylist}
%
%\section{applications}
%
%\begin{entrylist}
%  \entry
%    {2012}
%    {Who did I forget ?}
%    {\href{http://whodidiforget.com}{whodidiforget.com}}
%    {Guest list recommendation for Facebook events based on friends already attending the event.}
%  \entry
%    {2011}
%    {Fellows}
%    {\href{http://fellows-exp.com}{fellows-exp.com}}
%    {Automatic community detection among Facebook Friends in order to validate the \emph{cohesion} measure, creation of friend lists.}
%  \entry
%    {2008}
%    {Happy Flu}
%    {\href{http://happyflu.com}{happyflu.com}}
%    {Experiment aimed to measure viral spreading of content across the blogosphere.}
%\end{entrylist}

%\section{publications}
%Put your publications here!

%%% This piece of code has been commented by Karol Kozioł due to biblatex errors. 
% 
%\printbibsection{article}{article in peer-reviewed journal}
%\begin{refsection}
%  \nocite{*}
%  \printbibliography[sorting=chronological, type=inproceedings, title={international peer-reviewed conferences/proceedings}, notkeyword={france}, heading=subbibliography]
%\end{refsection}
%\begin{refsection}
%  \nocite{*}
%  \printbibliography[sorting=chronological, type=inproceedings, title={local peer-reviewed conferences/proceedings}, keyword={france}, heading=subbibliography]
%\end{refsection}
%\printbibsection{misc}{other publications}
%\printbibsection{report}{research reports}

\end{document}
